\section{Evaluation and Testing} % (fold)
\label{sec:evaluation_and_testing}
As mentioned in the analysis section\ref{sec:analysis} we are to evaluate our solution with white- and black box tests. Their description will thus be provided and followed by the results.

\subsection{White box tests}
The development of our solution was to iterate based on the behaviour of Raft. This included writing a test for each behaviour described in the condensed summary of Raft\cite{Raft}. For each of these tests its required functionality was then implemented in order to satisfy it.

%Forklar heraf resulterne af testene, og hvilke dele af koden der hører til hvilken 'behaviour'.

\subsubsection{Tests}
Although the functional tests, constructed during implementation, can be directly read in the code, a small example will be presented for completeness's sake.\\ \\
The following code snippet shows the test for the AppendEntries RPC attribute that a receiver should reply false if the given RPC call contains a term < currentTerm.
\lstinputlisting[caption=The test for the AppendEntries RPC Receiver implementation, language=JavaScript]{code_snippets/whitebox-test-example.js}

\subsection{Black box test scenarios}
Verifying that our solution actually is an implementation of Raft requires more in-depth tests in the form of validating that we uphold the properties that are argued to provide consensus. This is done by constructing a set of scenarios that reflect the behaviour of the components of Raft: Leader Election, Log Replication, and Safety. The scenarios are as following with success criteria and a simple use case:

\subsubsection{Tests}

\begin{itemize}
\item \textbf{Leader Election}:
    \begin{enumerate}
    \item A leader is elected upon initiate start up of the system.
        \begin{enumerate}
        \item The user starts the program.
        \item Server $S$ times out.
        \item $S$ converts to candidate.
        \item $S$ receives a majority vote and converts to leader.
        \end{enumerate}
    \item A new leader must be elected if the current leader fails.
        \begin{enumerate}
        \item The user starts the program.
        \item A server $L$ is elected as leader.
        \item The user crashes $L$ at run-time.
        \item Server $S$ times out and starts a new election.
        \end{enumerate}
    \end{enumerate}
\item \textbf{Log Replication}:
    \begin{enumerate}
    \item Log entry is replicated through the system.
        \begin{enumerate}
        \item The user starts the program.
        \item A server $L$ is elected as leader.
        \item The user requests that a value $x$ is replicated.
        \item $L$ replicates $x$ such that each log of each server contains $x$ in their log at the correct term and index.
        \end{enumerate}
    \item Awoken follower get its log updated.
        \begin{enumerate}
        \item The user starts the program.
        \item A server $L$ is elected as leader.
        \item The user crashes a server $S$.
        \item The user requests that a value $y$ is replicated.
        \item The user wakes up $S$.
        \item $L$ updates $S$'s log such that it contains $y$ in its log at the correct index and term.
        \end{enumerate}
    \end{enumerate}
\end{itemize}
\pagebreak
An example of a leader election tests looks as following:
\lstinputlisting[caption=The test for the leader election, language=JavaScript]{code_snippets/blackbox-test-example.js}
Here a cluster of 5 servers is set up in which the servers talk directly to each other without delay hence giving more reliable behaviour. After a short delay of 1 second the it is then assert that a leader should have been elected. The \texttt{assertOneLeader} function simply checks for all servers in the cluster that only one of them is in the leader state.

In appendix~\ref{sec:tests} a list of acceptance tests for the rest of the components of Raft are proposed, which are not implemented and run in our solution.
Please note that there is no acceptance test for state machine safety. This is because there's is no actual state machine in our implementation, thus testing this would be impossible. Further details about this in section~\ref{sec:discussion}.

And referring back to requirements in section \ref{sec:requirements}. The solution should also satisfy the requirements about availability and timing independence. The success criteria and scenarios for these requirements are as listed in appendix~\ref{sec:tests}.

The last scenarios about safety, availability, and especially timing independence are hard to formally verify since the program simulates a concurrent system, in which the utilisation of a formal verification tool would be necessary, which is out of scope of this project. A more detailed discussion of this will be done in section~\ref{sec:discussion}.

% section evaluation_and_testing (end)
