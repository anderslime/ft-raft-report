\setcounter{page}{1}

\section{Introduction} % (fold)
\label{sec:introduction}

This is the final report for the project in the course ``Fault tolerant systems'' (02228). The purpose of this project is to document our experience in implementing Raft - a consensus algorithm. Our approach to this is use the specification of the Raft paper~\cite[p.~4]{Raft} and iteratively construct a test for each behaviour the algorithm should have and implement its functionality that satisfies the given test.

IN this introductory section we will present he fundamental problem when talking about consensus in a distributed system with a basic description of the Raft algorithm.

\subsection{Problem}
Reaching consensus in a distributed system means that all processes in the network eventually agree on some value or state of the entire system. This is often needed when processes might be faulty thus bringing reliability and availability at stake on single-point-of-failure. Upholding these properties in a given distributed system then relies on the architecture utilised and hardware implemented in the processes.

A simple solution to this could be to initiate a vote among all correct processes on to what the value is in which the value is the result of the majority vote and thereby mask the faulty processes. The figure~\ref{consensus} below illustrates an example of a distributed system consisting of a number of notes. The value \textit{x} is what the system must agree upon. Though here we have a faulty process \textit{P1} which is responsible to return the result. Taking the fact of connectivity of system aside, the system will become unavailable because of the now faulty process.

\begin{figure}[h]
	\centering
	\includegraphics[width=0.5\textwidth]{consensus}
	\caption{A distributed system consisting of a number of nodes with a faulty one.}
	\label{consensus}
\end{figure}

The fundamental problem behind this is that you cannot rely on the individual processes to be reliable and thus solely store the value. The system needs more This boils down to a well-known problem - The Two Generals Problem.

% Måske burde the two generals problem komme her?

% Enig, har sat det ind her i stedet for.

\subsection{Motivation} % (fold)
\label{sub:motivation}

The elements of the problem presented serve as motivation behind consensus in a system of unreliable components. Because how do you rely on a value in a system, where some components will eventually be faulty? The basic problem of reaching consensus in a distributed network can be illustrated by the Two Generals Problem analogy.

\subsubsection{Two Generals Problem} % (fold)
\label{ssub:two_generals_problem}

In figure~\ref{generals} is an illustration of two generals from the same army who want to attack an enemy army and they have to attack at the same time in order to win. They cannot communicate directly to each other since they are at different fronts of the battlefield. In the context of a distributed system the generals here can be seen as two processes trying to agree on a value.
General 1 sends out a messenger to tell the second general that they should attack at dawn. The second general then receives this message, but the first general cannot be sure of this (the messenger might be captured or killed by the enemy on his way to the second general and vice versa). Again, in the context of distributed system, the unreliability of messenger can directly related back to the unreliability of message transmission in a normal distributed system.

\begin{figure}[h]
	\centering
	\includegraphics[width=0.8\textwidth]{twogenerals}
	\caption{Two generals tries to agree on when to attack the enemy by sending a messenger, but they are not sure the messenger survives his trip between their camps.}
	\label{generals}
\end{figure}

Now the second general sends the messenger back in order to acknowledge this, but the first general also has to acknowledge this, resulting in a never ending run for the poor messenger - thus the generals can never agree on when to attack the enemy.

\subsubsection{Properties of Distributed Consensus} % (fold)
\label{ssub:properties_of_distributed_consensus}

When want to define consensus, the goal is to satisfy a set of requirements i.e. properties that the distributed system must uphold. These are used to describe the systems fault tolerant features related to faulty processes. A faulty process can either fail by crashing or experience a Byzantine failure. Such a failure in the context of distributes system occur when e.g. a process for some reason transmits incorrect or malicious data throughout the network. Due to the arbitrary results of these kinds of failures, properties of a system are often distinguished by either tolerating them or not. A main difference in terms of properties of a system whether it tolerates Byzantine failures or not, is the validity and integrity. The integrity property for a system that does not tolerate Byzantine failures is as following~\cite{DistributedSystems}.

Byzantine failure can be related two the two generals problem in terms of the messenger being captured by the enemy and turned to spy on the generals i.e. given false information.

\begin{itemize}
\item non-Byzantine failure tolerant:
	\begin{itemize}
	\item \textbf{Validity}: If all processes propose the same value \textit{v}, then all correct processes decide \textit{v}.
	\item \textbf{Integrity}: Every correct process decides at most one value, and if it decides some value \textit{v}, then \textit{v} must have been proposed by some process.
	\end{itemize}
\item Byzantine failure tolerant:
	\begin{itemize}
	\item \textbf{Validity}: If all correct processes propose the same value \textit{v}, then all correct processes decide \textit{v}.
	\item \textbf{Integrity}: If a correct process decides \textit{v}, then \textit{v} must have been proposed by some correct process.
	\end{itemize}
\end{itemize}

The clearest commonality here, is that you should always be able to say that all correct processes must be able to decide on the same value or state, as mentioned earlier. Though the main difference is that with Byzantine processes, you must be able to say if they are all correct before stating they can derive the same value. This fact differentiates many solutions to this problem, depending on which property set they can satisfy.

Also, as discussed in \cite{Raft} a consensus algorithm for a non-Byzantine system has the following properties: safety, availability, timing interdependency, and majority vote on procedure calls. The safety property can be directly related back to our validity and integrity properties, as the system's safety and availability is measured be the correctness of return result upon a request.

% subsubsection properties_of_distributed_consensus (end)

% section introduction (end)
