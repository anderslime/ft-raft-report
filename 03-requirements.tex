\section{Requirements} % (fold)
\label{sec:requirements}
The tool we are implementing should serve as an illustration tool, to show how the Raft algorithm works for given scenarios of faulty processes in a distributed system.
The requirements are split into two different aspects i.e. functional and quality attributes:
\subsection{Functional requirements}
The end product must satisfy the following functional requirements given the MoSCoW method:
\begin{enumerate}
\item The tool must illustrate a scenario of a given set of processes in a distributed system, in which a leader is elected.
\item The tool must be an implementation of the Raft algorithm as specified in \cite{Raft}.
\item The user should be able to input parameters to vary the scenario, such as the number of processes.
\item The user must then be able to disconnect any process.
	\begin{enumerate}
	\item If a leader is disconnected a new leader must be elected automatically.
	\end{enumerate}
\item The user must be able to request that a log entry is replicated to the system at run time.
\item The tool could be further extended with a visualisation of the given distributed system.
\end{enumerate}
\subsection{Quality requirements}
The quality requirement of the illustration tool are inherited from the properties that are provided by the Raft algorithm.
	\begin{enumerate}
	\item The Raft implementation should have the following properties: Safety, availability, timing independence, and that commands complete as soon as a majority has responded.
		\begin{enumerate}
		\item Safety: There must always be at most one leader in the distributed system.
		\item Availability: The system should be available as long it is possible to elect a leader through a majority vote.
		\item Timing independence: Safety should not depend on the timing of the system, i.e. the speed of a given process should provide it with an advantage in terms of the consensus of the system. For this it is specified that:
		\begin{center}
		$broadcastTime << electionTimeout << MTBF$ \cite{Raft}.
		\end{center}
		\end{enumerate}
	\end{enumerate}
% section requirements (end)
