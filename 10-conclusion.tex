\section{Conclusion} % (fold)
\label{sec:conclusion}
The end product simulates a cluster of servers, where each server can be crashed and later restarted. The user can also request that a value is replicated throughout the system, after which the servers should reach consensus on this i.e. a majority of servers should return a successful log replication. This fault tolerant feature is provided by giving each server a Raft implementation.\\ \\
The implementation of Raft was done in Javascript, which provided us with a much deeper understanding of Raft and the consensus problem and its possible solution. An important lesson learned was that the choice of tool used for the job, i.e. simulating a concurrent system and verification its correctness, serves an important role both in terms of implementation overhead and verification.\\ \\
Developing the solution was done by adopting the behaviour that is specified in the Raft paper, and for each of these behaviour constructing a test that should then be satisfied before moving on to the next behaviour. Practicing this behavioural development process allowed us to structurally implement Raft without having to give much thought to the overall design.
% section conclusion (end)
